\title{搜索引擎技术基础——wikirank实验
}
\author{MaskRay}
\maketitle

\section{输入和数据存储}

用\texttt{getline}读入\texttt{node.map.utf8}文件的每一行,把其中的名字和编号都映射到一个本地编号,
本地编号是从0开始连续的,从而可以使用\texttt{vector}存储PageRank而无需使用关联数组,可以加快计算。

\texttt{wiki.graph}中边的表示方式和forward star representation很像,因此使用这个方式存储边。
顶点\texttt{u}的出边表示为\texttt{ends[from[u]...to[u]]},这样存储边很紧凑,内存使用量小。

\section{算法实现}

\begin{enumerate}
  \item 初始化\texttt{rank[i] = 1.0/n, i=0...n-1}
  \item (迭代开始) \texttt{deadend = 1.0/n},\texttt{rank2 = 0, i=0...n-1}
  \item \texttt{rank2[v] += rank[u]*(1-residual)/outDegree[u], for (u,v)}
  %\item 若顶点\texttt{u}出度为0则\texttt{deadend += residual*rank[u]}
  \item (迭代结束) 把\texttt{rank2}复制到\texttt{rank}
\end{enumerate}

这个实现考虑到了出度为0的顶点,把它们的未分配出去的PageRank都聚合到\texttt{deadend}上,
然后再让\texttt{deadend}把PageRank均匀地分配给各个顶点。

\section{构建和运行}

使用C++ 11编写,可以用\texttt{g++ -std=c++11 -O3 main.cc}编译。

\begin{verbatim}
% ./main -h
Usage: ./main [OPTIONS] node_map graph

Options:
 -i, --niter NITER     number of iterations (default: 30)
 -r, --residual NITER  residual probability (default: 0.85)
 -s, --scc             strongly-connected components
 -w, --wcc             weakly-connected components
\end{verbatim}

程序接受命令行选项指定迭代次数和系数residual probability。

\begin{verbatim}
% for i in $(seq 4); do echo --$i-- && ./main -i $i /tmp/nmap /tmp/graph; done
--1--
b       0.380556
c       0.380556
a       0.238889
--2--
b       0.367176
c       0.367176
a       0.265648
--3--
b       0.370967
c       0.370967
a       0.258066
--4--
b       0.369893
c       0.369893
a       0.260215

% ./main -i 50 -r 0.9 /tmp/nmap /tmp/graph
b       0.371795
c       0.371795
a       0.256410
\end{verbatim}

\texttt{icpc -O3}编译的程序,迭代10次,
在我的\texttt{Intel(R) Core(TM) i5-3317U CPU @ 1.70GHz}笔记本上用时53.84秒。

\section{样例输入集}

\texttt{node.map.utf8}共3722312行,即有$n=3722312$个顶点。
\texttt{wiki.graph}有1109153行,其中有7358个重复行,不重复的行有1101795个。
使用如下命令即可生成不重复的数据集:

\begin{verbatim}
awk '!a[$0]++' wiki.graph > uniq.wiki.graph
\end{verbatim}

\subsection{出度分析}

如下命令可以生成出度的frequency histogram:

\begin{minted}{bash}
awk -F'[:,]' '{out[NF-2]++}END{for(i in out)if(i>0)print i,out[i]}' uniq.wiki.graph | sort -n > out.degree.dat
\end{minted}

\begin{figure}[H]
  \centering
  \includegraphics[width=0.8\textwidth]{img/outd.png}
  \caption{出度分布}
\end{figure}

对两个轴分别取对数后数据点基本在一条直线上,但出度较小时拟合情况不好。

\begin{verbatim}
> summary(lm(log(number) ~ log(outdeg), data=a))

Call:
lm(formula = log(number) ~ log(outdeg), data = a)

Residuals:
    Min      1Q  Median      3Q     Max 
-3.8569 -0.3878  0.0841  0.4221  3.1858 

Coefficients:
            Estimate Std. Error t value Pr(>|t|)    
(Intercept) 15.26071    0.09438   161.7   <2e-16 ***
log(outdeg) -2.04316    0.01436  -142.3   <2e-16 ***
---
Signif. codes:  0 ‘***’ 0.001 ‘**’ 0.01 ‘*’ 0.05 ‘.’ 0.1 ‘ ’ 1

Residual standard error: 0.626 on 1619 degrees of freedom
Multiple R-squared:  0.926,     Adjusted R-squared:  0.9259 
F-statistic: 2.025e+04 on 1 and 1619 DF,  p-value: < 2.2e-16
\end{verbatim}

\subsection{入度分析}

如下命令可以生成入度的frequency histogram:

\begin{minted}{bash}
awk -F'[:,]' '{for(i=2;i<=NF;i++)o[$i]++}END{for(i in o)x[o[i]]++;for(i in x)if(i>0)print i"\t"x[i]}' uniq.wiki.graph | sort -n
\end{minted}

\begin{figure}[H]
  \centering
  \includegraphics[width=0.8\textwidth]{img/ind.png}
  \caption{入度分布}
\end{figure}

\subsection{强连通分量}

使用Tarjan's SCC算法求强连通分量。代码中使用了一个技巧,即把\texttt{lowlink}数组实现为函数的返回值,
从而可以少使用一个数组。因为使用了递归函数,运行时需要调整栈大小以避免栈溢出导致segmentation fault:

\begin{minted}{bash}
ulimit -s unlimited
./main -s node.map.utf8 uniq.wiki.graph
\end{minted}

程序会打印出各个强连通分量的大小。
如下命令可以生成强连通分量统计信息,左列为数目,右列为强连通分量大小(顶点数):

\begin{verbatim}
% ./main -s node.map.utf8 uniq.wiki.graph | sort -n | uniq -c
3120467 1
   3869 2
    727 3
    303 4
    127 5
     79 6
     47 7
     50 8
     28 9
     20 10
     13 11
      9 12
     11 13
      6 14
      2 15
      6 16
      5 17
      3 18
      3 19
      1 20
      1 21
      2 22
      1 24
      1 26
      1 28
      1 29
      1 43
      1 49
      1 56
      1 59
      1 67
      1 105
      1 156
      1 586897
\end{verbatim}

\begin{itemize}
  \item 最大的强连通分量大小为586897,占总数$586897/3722312=15.7670\%$的页面在这个强连通分量中。
  \item 有大量大小为1的强连通分量,这部分页面占$3120467 / 3722312=83.8314\%$。
\end{itemize}

\subsection{弱连通分量}

为了减少空间占用,弱连通分量通过union-find实现,
并且采用了weighted quick-union的技巧,合并集合时通过集合大小启发式选择父顶点,
集合大小的信息和表示父顶点的信息合并存储在一个数组元素里,节省了内存。
Find使用了path halving,
把查找链上一半的顶点从指向父顶点改为指向祖父顶点。
使用了这两个技巧仍然能保证某个Ackermann逆函数的时间复杂度,但实现更为简洁,且没有使用递归。

如下命令可以生成弱连通分量统计信息,左列为数目,右列为弱连通分量大小(顶点数):

\begin{verbatim}
% ./main -w node.map.utf8 uniq.wiki.graph | sort -n | uniq -c
   2367 1
    471 2
     48 3
     16 4
      5 5
      4 6
      1 8
      1 12
      1 35
      1 3718691
\end{verbatim}

\begin{itemize}
  \item 最大的弱连通分量大小为3718691,$3718691/3722312=99.9027\%$的页面在这个弱连通分量中。
  \item 弱连通分量比强连通分量少很多。
\end{itemize}

